If we treat a wooden sample as an infinite body (i.e. ignoring edge effects), composed of an orthotropic homogeneous hygroelastic material, with no intra-body gradients (i.e. a quasi-static process, only ever infinitesimally far away from equilibrium), subjected to an internal hydrostatic water-tension load, then the surface strains would be given by

\begin {eqnarray}
\varepsilon_T &=(S_T+S_{TZ}+S_{TL}) \sigma +\varepsilon^o_T\\
\varepsilon_L &=(S_{TL}+S_{ZL}+S_L) \sigma +\varepsilon^o_L\\
\gamma_{TL} &=0
\end{eqnarray}
If the material is much stiffer in the longitudinal direction than in any transverse direction then $S_L$≈0
and $S_{TL}$≪$S_T$+$S_{TZ}$

. Then, when the above becomes:

$$\varepsilon_T \varepsilon_L = (S_T+S_{TZ})\sigma + \varepsilon^o_T = \left(S_{TL}+S_{ZL}\right) \sigma + \varepsilon^o_T∘$$

Previously $\varepsilon_T$ and $S_T$ have been measured and $\sigma$ estimated assuming $S_{TZ}=0$ [bookerXX]. Because εL is also related to σ it seemed worthwhile trying to use DIC to measure both it and εT in order to better understand the stress-strain state in the early stages of drying and to develop better estimates for σmax

(a wood quality measure that might potentially contribute to explaining why some material is susceptible to intra-ring checking during drying and other material is not).

In reality:

   \begin{itemize}
\item  edge effects are present,
\end{itemize}
\begin{itemize}
\item     the material is not homogeneous,
\end{itemize}
\begin{itemize}
\item     intra-body gradients in moisture content and water tension do exist,
\end{itemize}
\begin{itemize}
\item     the surface strains are not measured exactly in the material principal directions,
\end{itemize}
\begin{itemize}
\item     the material is not perfectly orthotropic,
\end{itemize}