Previously $\varepsilon_T$ and $S_T$ have been measures and $\sigma$ estimated assuming $S_{TZ}=0$\citep{bookerXX}. Because $\varepsilon_L$ is also related to $\sigma$ it seemed worthwhile trying to use DIC to measure both it and $\varepsilon_T$ in order to better understand the stress-strain state in the early stages of drying and to develop better estimates fro $\sigma_\max$ (a wood quality measure that might potentially contribute to explaining why some material is susceptible to intra-ring checking during drying and other material is not.

\begin{eqnarray}
\varepsilon_T &= \left(S_T + S_{TZ}\right)\sigma + \varepsilon^o_T \\
\varepsilon_L &= \left(S_{TL} + S_{ZL}\right)\sigma + \varepsilon^o_L \\
\gamma_{TL} &= 0 \\
\end{eqnarray}
If the material is much stiffer in the longitudinal direction than in any transverse direction then $S_L$ ≈ 0 and $S_{TL}
$≪$S_T$+$S_{TZ}$




In reality:

    edge effects are present,
    the material is not homogeneous,
    intra-body gradients in moisture content and water tension do exist,
    the surface strains are not measured exactly in the material principal directions,
    the material is not perfectly orthotropic,